\documentclass[UTF8]{ctexart}
\usepackage{graphicx}
\title{全面分析:家庭物联网设备安全调研}
\author{Deepak Kumar  Kelly Shen  Benton Case  Deepali Garg\\Galina Alperovich  Dmitry Kuznetsov  Rajarshi Gupta  Zakir Durumeric \\Stanford University  Avast Software University of Illinois Urbana-Champaign}
\date{\today}
\begin{document}
	\tableofcontents
	\maketitle
	\section*{摘要}
	在本文中,我们通过利用从一千六百万个家庭中八千三百万个设备上由用户发起的网络扫描中所收集来的数据,提供了第一个大规模的对现实生活家庭中物联网设备的实证分析。我们意识到,物联网设备的使用已经相当普遍:在各大洲,超过半数的家庭至少拥有一个物联网设备,其中设备的种类和用户对厂商的选择在不同的地区有着显著差异。比如,尽管有接近一半的北美家庭拥有联网的电视或是其他流媒体设备,但在南亚,大多数设备却是监控摄像头,电视之类所占比例只有不到百分之三。我们研究了这些物联网设备的安全态势,并详细阐述了它们的开放服务,可能的弱默认凭证以及针对已知攻击存在的弱点。此外,我们还发现设备的安全性就如其使用一样存在地域差异,即使是对于特定厂商的产品而言也是如此。比如,虽然在北美只有不到百分之十七的TP-Link家用路由器存在容易猜测的口令,在东欧和中亚却有近一半设备存在这样的弱口令。最后,虽然对大多数家庭来说,其使用的某些设备类型可能比较小众而并没有被深入地研究,但我们希望通过对这一复杂生态的阐释,能够帮助安全社区开发出广泛适用于当今家庭环境的解决方案。
	
	\section{引言}
	由于许多流行物联网设备的安全态势薄弱,从而使得攻击者可以对其展开大量的攻击继而造成严重的后果。另一方面,尽管在安全社区中并不缺乏对于物联网设备的关注,但是却几乎没有关于消费者使用的设备类型以及用户在实际应用中如何进行设备配置的相关调查。所以我们在这项工作中提供了一个对于一千六百万个家庭中八千三百万个物联网设备的大规模实证分析。我们与Avast Software——一家杀毒软件公司进行合作,该公司提供的安全软件允许客户扫描他们所在的本地网络以发现支持弱认证或存在远程利用漏洞的物联网设备。利用从一千六百万个同意为了研究和开发目的而共享个人信息的家庭所发起的扫描中收集到的数据,我们描绘了目前物联网设备的安全图景以及它们的安全态势。
	
	物联网设备的使用是广泛的。在全球有三个地区超过一半的家庭至少拥有一个物联网设备,并且在北美,有超过百分之七十的家庭拥有联网设备。像是智能电视之类的流媒体设备在十一个国际地区中的七个是非常常见的,但是共性之外也存在一些显著差异。比如,监控摄像头在南亚及东南亚是最受青睐的,而家电用品则在东亚和撒哈拉以南的非洲大受欢迎。Home assistant智能家居系统纵使在北美有超过百分之十的家庭使用,但是在全球其他市场却少见使用。此外,值得一提的是,虽然世界上有着总共达一万四千家的物联网设备制造商,但意外地,我们发现全世界范围内百分之九十的设备都只是来源于其中的一百家。少数的公司,像是苹果,惠普和三星在世界范围内处于统治地位,但同时也存在一些小型的地域性供应商。比如伟视达——一个土耳其制造商,其虽然是北非和中东地区的第三大多媒体设备供应商,但是也仅局限于这一片地区,在更大范围内的供应量可以忽略不计。
	
	至今为止,仍存在惊人数目的设备支持带有弱凭证的FTP和Telnet协议,其中在撒哈拉以南的非洲,北非,中东以及东南亚,大约有一半的设备支持FTP,并且在中亚有接近百分之四十的家用路由器使用Telnet。就如设备类型以及厂商选择存在地域差异一样,弱凭证的使用也存在显著性的地方差异。比如,虽然在欧洲和大洋洲只有不到百分之十五的支持FTP的设备允许弱身份认证,但在东南亚及撒哈拉以南的非洲却有一半以上的设备存在这样的问题。有趣的是,这并不完全是厂商的问题,因为尽管在北美只有不到百分之二十的TP-Link家用路由器允许使用弱口令访问管理界面,在东欧,中亚以及东南亚却有接近一半的设备存在这样的问题。目前在我们的数据集中,就有接近百分之三的家庭中的设备存在已知的漏洞或弱口令。
	
	我们的调查结果指出,物联网并不是一个未来的安全问题,而是一个当今的安全问题。我们认为,在世界各地的家庭中已经存在由各类网络互联嵌入式设备所构成的相当复杂的生态了,但我们现在讨论的物联网设备却和最近的研究工作所考虑的问题具有较大不同。我们希望通过对消费者所购买的设备类型的调查展示,使得安全社区能够开发出较为有针对性的适用于今天家庭环境的解决方案。
	
	\includegraphics[width=.9\textwidth]{1.jpg}
	
	(a)数据共享声明 \quad (b)WiFi Inspector界面 (c)WiFi Inspector初始化
	
	图1:\textbf{WiFi Inspector}允许用户扫描他们所在的局域网络来发现存在安全隐患的物联网设备。在软件安装过程中,会明确告知用户其数据会被收集用以研究目的。为方便阅读,我们将相关内容展示在了附录A中。
	
	\section{方法与数据集}
	我们的研究基于从Avast收集到的数据,被动的network      telescope(网络流量监控系统)以及主动的因特网范围内的扫描。在这一部分,我们将探讨这些数据集以及它们在我们的分析中所扮演的角色。
	
	\subsection{WiFi 督察}
	Avast  Software是一家提供杀毒和各类安全产品的软件公司,像是Avast Free Antivirus便是其中之一。这家公司的软件是以免费增值模式销售的,即免费提供产品的基础版本,而对进一步的附加服务进行收费。Avast提供的软件在大量机器上得以使用,预估在一亿六千万的Windows系统和三百万的Mac     OS系统上都装有Avast公司提供的安全软件,并且其拥有大约百分之十二的杀毒软件市场占有率。
	
	截止到2015年,Avast的所有安全产品包括一个名为WiFi Inspector(WiFi 督察)的工具持续助力用户为家庭网络中的物联网设备及其他计算机提供安全保护。WiFi Inspector在用户的个人电脑上本地运行,它能够对本地子网进行扫描来发现接受弱凭证和存在可远程利用漏洞的设备。扫描也能够由终端用户手动发起。WiFi Inspector会在主界面(图1)告知用户它在扫描期间所发现的安全问题,并提供一个打上标签的物联网设备及其存在的漏洞的清单。接下来我们介绍一下WiFi Inspector是如何工作的。
	
	\subsubsection*{网络扫描}
	为了提供上述的本地网络设备清单,WiFi Inspector首先会根据本地的ARP表以及通过主动的ARP,SSDP和mDNS扫描来生成一列待扫描对象。接着,它会按照IP地址递增的顺序利用ICMP来检测目标设备的常见TCP/UDP端口的监听服务\footnote{WiFi Inspector会扫描一组TCP/UDP端口:常用TCP端口(如80,443,139,445);与安全相关的TCP端口(如111,135,161);常用UDP端口(如53,67,69);以及与提供用来给设备分类的数据的服务相关的端口(如20,21,22)。当主机及时响应后,探测器还会进一步探测另一组不那么常见的端口(如81-85,9971)。只要设备联网,则扫描器就可以进行识别,同时根据目标主机性能的不同,扫描器将最多扫描200个端口。}。扫描过程会在完成本地网络扫描或是超时发生之后终止。在这个发现设备的过程完成后,扫描器就开始尝试收集应用层数据(如:HTTP根目录页面,UPnP根设备描述文件和Telnet的横幅(banner)信息)
	
	\subsubsection*{检测设备类型}
	为了给用户提供一个易读的主机列表,WiFi Inspector会对扫描中收集到的应用层和传输层数据执行分类算法,从而将设备划分到以下十四个种类的其中之一:
	
	\noindent
	1.计算机\\
	2.网络节点(例如:家用路由器)\\
	3.移动设备(例如:iPhone 或 Android手机)\\
	4.可穿戴设备(例如:Fitbit,Apple Watch)\\
	5.游戏机(例如:XBox)\\
	6.智能家居(例如:Nest 恒温器)\\
	7.存储设备(例如:家用NAS——网络附属存储)\\
	8.监控摄像头(例如:IP camera)\\
	9.办公设备(例如:打印机或扫描仪)\\
	10.家庭语音助手(例如:Akexa)\\
	11.交通工具(例如:Tesla)\\
	12.电视(例如:Roku)\\
	13.家用电器(例如:智能冰箱)\\
	14.其他(例如:智能牙刷)
	
	\resizebox{\textwidth}{20mm}{
	\begin{tabular}{|c|c|c|c|c}
		\hline
		协议& 域& 搜索模式& 标签& 可信度\\
		\hline
		DHCP& 类ID& $(?i)SAMSUNG[- :_]Network[- :_]Printer$ & 打印机& 0.90\\
		UPnP& 设备类型& $.*hub2.*$& 物联网集线器& 0.90\\
		HTTP& 标题域& $(?i)Polycom - (?:SoundPoint IP )?(?:SoundStation IP )?$& ip电话& 0.85\\
		mDNS& 域名&$(?i)_nanoleaf(?:api|ms)?\._tcp\.local\.$& 电灯 & 0.90\\
		\hline
	\end{tabular}}

	\begin{center}
		表1:\textbf{设备分类规则举例}——我们的设备标记算法集合了利用网络层和应用层数据进行分析的1000条专家规则和一个监督分类器。在此,我们展示了能够覆盖1000个随机设备样本中百分之六十的专家规则的其中一小部分。
	\end{center}
	
	在后文中,我们将主要考虑上述提到的后11种设备。同时由于分类器会在很大程度上影响最终的结果,所以我们将特别在2.2节中阐述分类算法的细节。
	
	\subsubsection*{制造商标签}
	为了给各类设备打上标签,WiFi Inspector会将设备类型与其相应的制造商联系起来(比如任天堂游戏机)。Avast会从每一个设备的前24位MAC地址即OUI(由IEEE分配的组织唯一标识符,用来区分不同的厂家)来得知该设备的制造商。然而我们注意到,有时获得的MAC地址是网络接口提供商的地址而不是我们需要的设备制造商的地址。比如与索尼游戏机相对应的MAC地址实际是属于两个电子元器件厂商FoxConn或AzureWave的地址。在本工作中,我们将手工解决并记录上述问题。
	
	\subsubsection*{检查弱凭证}
	WiFi Inspector会针对FTP和Telnet服务以及基于身份认证的使用HTTP协议的网页进行字典攻击,以此发现能利用弱凭证来认证的设备。如果可能的话,WiFi Inspector还会尝试使用大约200个由已知默认值(如admin/admin)和来自各类常用口令列表中的字符串(如user,1234,love)来登录到它识别到的基于HTTP的管理界面。若是有所发现,WiFi Inspector会立即告知用户存在弱口令的设备。
	
	\subsubsection*{检测常见的漏洞}
	除了检查弱凭证之外,WiFi Inspector还会在不危害目标设备的前提下对最近公布的大约50个漏洞进行检测(如,CVE-2018-10561,CVE-2017-14413,EDB-ID-40500,ZSL-2014-5208,和 NON-2015-0211)。由于我们更倾向于对那些更加广为人知,且使用更为广泛的厂商设备的扫描,所以我们最后并没有提供基于发现的漏洞的各厂商之间的比较。
	
	\subsection{设备识别算法}
	在我们的工作中相当重要的一个环节就是对家用物联网设备的厂商及其类型的识别,接下来我们便会在本节中描述由Avast开发的这一套算法。
	
	\subsubsection*{分类器}
	WiFi Inspector基于针对网络数据和应用层数据的一系列专家规则以及一个监督分类算法来对设备类型进行标记。由于制造商通常会在设备的网页管理界面和FTP,Telnet的横幅上包含关于设备型号的一些信息,因此分类通常是可行的。除此之外,设备本身还会通过UPnP和mDNS来广播一些细节信息。WiFi Inspector使用专家规则——正则表达式来解析一些字段(比如telnet 横幅或HTML 标题),从而给相应的设备打上标记以此表明它们的厂商和型号。这种方法虽然可能不具有普适性,但却能可靠地识别比较常用的设备。WiFi Inspector包含能够识别近200家厂商的设备的大约1000条专家规则,我们在表1中展示了这些规则的一小部分。然而,这些规则只能识别1000个手工标记的随机样本中的百分之六十,为了对剩余的设备进行分类,WiFi Inspector会综合考虑4个分别利用网络层数据,UPnP响应,mDNS响应以及HTTP数据单独进行分类的监督学习分类器的结果。因此,当要识别一个设备时,WiFi Inspector首先会使用专家规则进行匹配,若是没有匹配项,则会进一步应用四个监督分类器来做识别。
	
	网络分类器使用随机森林来构建的,它集合了一个设备的下列网络特征:\\
	1.MAC地址\\
	2.本地IP地址\\
	3.监听服务(即端口和协议)\\
	4.各个端口上的应用层响应\\
	5.DHCP的类编号和主机名\\
	UPnP,mDNS和HTTP分类器收集原始文本响应,将每一次响应都使用词袋模型来处理,并利用TF-IDF(词频-逆向文件频率)来对所有响应中得到的词进行加权。最后结果会作为输入提供给朴素贝叶斯分类器。若是没有匹配项,则会进一步应用四个监督分类器来做识别。
	
	\subsubsection*{训练和评估}
	为了训练上述提到的监督算法,Avast从用户发起的扫描中搜集了大约五十万个随机设备的数据。其中二十万个设备是通过迭代聚类/标记过程手工分类的,在此,研究者们基于网络属性和已标记簇筛选和再聚类直到所有设备都被标记。其余的三十万个设备则是利用专家规则来标记的。为了调整模型参数,我们在原始训练集上执行了5次交叉验证。然而,由于初始聚类用于在聚类/标记过程中帮助识别设备,因此这些数据集并不用于验证。但是Avast另外筹备了1000个手工标记设备的测试集,而这些数据并没有被用于训练。最终分类器呈现出百分之九十六的准确性,百分之九十二的覆盖率,宏平均0.80的F1分数(表2)。最后,我们将无法识别的设备标记为“未知”。
	
	\setlength{\tabcolsep}{5mm}{
	\begin{tabular}{|c|c|c|c}
		\hline
		分类器& 覆盖范围& 准确性& 宏观F1\\
		\hline
		集成监督算法& 0.91& 0.95 & 0.78\\
		网络层数据分类& 0.89& 0.96& 0.79\\
		UPnP相应分类& 0.27& 0.91& 0.37\\
		mDNS相应分类& 0.05& 0.94& 0.25\\
		HTTP数据分类& 0.14& 0.98& 0.23\\
		最终的分类器& 0.92& 0.96& 0.80\\
		\hline
	\end{tabular}}

	\begin{center}
		表2:\textbf{设备分类性能}——最终的分类器结合了监督分类器和专家规则,并且对手工标记的含1000个设备的设备集达到了百分之九十二的识别覆盖率和百分之九十六的准确性。
	\end{center}
	
	\begin{tabular}{c|c|c}
		\hline
		地区& 家庭& 设备\\
		\hline
		北美& 1.24 M (8.0\%) & 9.2 M (11.1\%)\\
		南美& 3.2 M (20.9\%) & 18 M (21.6\%)\\
		东欧& 4.2 M (27.2\%) & 18.8 M (22.6\%) \\
		西欧& 2.9 M (19.1\%) & 15 M (18.0\%) \\
		东亚& 543 K (3.5\%)  & 3 M (3.7\%) \\
		中亚& 107 K (0.7\%)  & 500 K (0.6\%)\\
		东南亚& 813 K (5.3\%)& 3.6 M (4.3\%) \\
		南亚& 824 K (5.3\%)  & 6.6 M (7.7\%) \\
		北非,中东& 1.2 M (7.5\%) & 6.1 M (7.3\%) \\
		大洋洲& 124 K (0.8\%)&680 K (0.8\%)\\
		撒哈拉以南非洲& 266 K (1.7\%) & 1.8 M (2.2\%)\\
		\hline
	\end{tabular}

	\begin{center}
		表3:\textbf{家庭地区分布}——数据库中一千五百五十万个家庭以及八千三百万个设备的数据来自于不同的地理区域。因为这个分类反映的是Avast在不同地区所占的市场份额,而不是实际各地区家庭或设备的密度,所以我们的分析只会针对单个地区进行。
	\end{center}

	\subsection{Avast数据集}
	Avast会从同意信息共享的WiFi Inspector用户处收集设备,漏洞以及弱凭证等信息用于研究,而用户在安装过程中会被明确告知信息收集这一事实(表1),若不同意,可以选择不参与信息收集。我们和Avast联合对各个地区收集到的数据进行分析,为了尽可能保障用户隐私,我们团队并没有共享任何个人记录或个人身份信息。虽然WiFi Inspector提供自动的漏洞扫描,但在这项工作中我们只会从用户自行发起的扫描中搜集数据,以保证WiFi Inspector是在用户知情的条件扫描他们的网络。另外,我们只对被标记为Windows家庭网络的网络进行分析,而将公共网络排除在外。在2.6节,我们会详述有关数据收集方面存在的道德问题以及我们的保障措施。
	
	我们特别分析了在2018年12月1日至31日在Windows上安装运行所得到的设备扫描数据,这一数据集里面包含了来自241个国家和地区一千五百五十万个家庭中的八千三百万个设备以及14300个不同制造商的数据。对于在这一时期的安装过程中进行了多次扫描的,我们使用发现设备数量最多的且最近的一次扫描结果。我们将各个国家按照ISO 3166-2标准划分到了11个地区中,如表3所示,WiFi Inspector在欧洲和南美要比在北美的受众更广。由于不同的市场占有率以及物联网设备部署的显著地区差异,我们将分区域进行讨论。
	
	\subsubsection*{有效性存疑}
	虽然WiFi Inspector被大量的家庭所使用,但是数据集仍然可能出现一些偏差。首先,这些数据是在以用户的电脑上安装了杀毒软件为基础的前提下进行搜集的,然而很少有研究表明使用杀毒软件的用户会有更多或更少的安全意识。其次,由于不同操作系统的差异,我们仅仅分析从Windows系统上得到的数据,而这可能会使得我们研究的家庭偏向不同的社会经济群体或引入其他一些偏差。第三,WiFi Inspector会主动通知用户它所发现的问题,而这会使得一部分用户给自己的机器打上补丁,修改默认的口令,但同时会让另一部分认为麻烦的人直接退货。这就会使得这项研究中涉及的家庭要比实际情况下更加安全从而得出一个具有误导性的结论(因为退货的人显然不在研究范围内)。
	
	\subsection{网络流量监控}
	虽然WiFi Inspector可以扫描得到家庭网络中存在的设备类型,但凭这些信息并不足以得出相应设备是否被入侵的事实。为了了解设备是否已被感染并通过扫描进一步去入侵其他设备(如Mirai[4]中所述),我们考虑IP地址扫描是在一个由大约470万个IP地址组成的大型网络中进行的。我们特别对2019年1月1日24小时内产生的扫描流量使用Durumeric等人讨论的方法进行了分析:我们认为如果一个IP地址在一个480秒的窗口中通过同一端口与该网络中的至少25个其他不同设备存在连接,那么该设备正在执行扫描。在测量期间,我们总共从14亿个包中的529000个IP地址上观察到了170万次扫描。当天,在WiFi Inspector扫描的500716个家庭中,发现1865(百分之0.37)个家庭中存在对目标网络进行扫描的设备。
	
	\subsection{互联网范围的扫描}
	更进一步地,我们在已有的WiFi Inspector收集到的数据之外,再加入通过Censys扫描整个互联网所得到的数据来证实网关(如家用路由器)上存在的漏洞是否会被远程利用。并且我们交叉对比了Censys和Avast在2019年1月30日的24小时之内的数据来发现潜在的DHCP扰动。我们还检测了接受弱凭证认证的设备是否在公共IP地址上存在登录接口。这部分结果我们将在第4部分讨论。
	
	\subsection{伦理道德方面的考虑}
	WiFi Inspector会从用户的家庭收集数据,为了保证这些数据确实是按照用户的期望进行收集的,我们只会对那些明确同意为了研究目的而共享数据的家庭进行数据收集。而这份数据共享协议并没有藏在什么偏僻角落,而是白底黑字清清楚楚地展示在了用户面前,我们在图1中已经展示了在安装过程中用户对此进行确认的对话框,且用户可以自行选择不参与。同时,数据共享协议是在程序主窗口出现前最后一条展示给用户的信息,意味着用户若是想要关闭数据收集,不用自己一直惦记着,程序本身就会提醒用户。
	
	为了能够保持家中设备的最新信息,WiFi Inspector会周期性地运行,自动扫描本地网络。但是这些自动扫描并不会执行任何漏洞测试或是弱口令检查的工作,它们只是根据banners和MAC地址来识别设备而已。我们只会对明确是用户手动发起的网络扫描进行数据搜集与分析。
	
	为了保护用户的隐私以及最小化用户的风险,Avast只会与我们团队共享汇总之后的数据。这份数据集合了设备制造商,地区以及设备类型信息。举个例子,我们团队只知道其中最小的区域包含10万个家庭,而无权对单独的某一个家庭或用户的数据进行访问的,并且我们也没有共享任何用户个人身份信息。同时,在这项工作中,Avast没有再收集其他任何信息,也没有改变任何原始数据的保存期限。最后,任何出现在了本文中的数据都不会被长期存储。
	
	在内部,Avast遵守严格的隐私政策:即所有数据均匿名,且任何个人身份信息都不会跟外部研究人员共享。所有对WiFi Inspector收集数据进行的处理都满足个人数据保护条例,如GDPR(欧盟通用数据保护条例),并且对欧盟以外的区域也适用。特定的标识像IP地址之类的信息,依照GDPR的规定会被删除,并且仅当为了产品的安全功能明确需要时才会进行收集。
	
	\includegraphics[width = .8\textwidth]{2.jpg}
	\begin{center}
		图2:\textbf{不同地区的设备}——在不同的地区,各类设备的使用存在显著差别。目前设备数量所占比例最高的是北美,平均每户家庭拥有七台主机。然而,南亚每户家庭则平均只有两台。统计的设备数量是从2开始的,因为每个家庭至少要有1台电脑和一台路由器才会被纳入我们的统计范围内。
	\end{center}

	\section{3}
	
	\section{4}
	
	\section{5}
	
	\section{相关工作}
	我们的工作建立在多个领域的研究基础之上,其中主要是家庭网络监测和物联网安全。
	
	\subsubsection*{家庭网络检测} 
	家庭网络监测的早期研究主要集中在调式网络上——像Netalyzer[35]之类的项目被构想成能够允许用户自己对他们的家庭网络连接进行调试[9,15,49]。后续出现了许多使用类Netalyzer扫描来调查家庭网络设备状态的论文[1,9,14],其中甚至还有一些研究试着了解联网设备对家庭用户行为的影响[10]。
	
	与我们的工作最接近的是由Grover等人呈现的。他们在21个国家的100个家庭中安装了带有自定义固件的家庭路由器,以此来测量家庭网络的可用性,基础设施以及网络的使用情况。他们的工作侧重于家庭网络的总体网络属性,同时也可以基于这些路由器在网络中的位置来监测网络。相反,我们工作侧重于NAT背后的设备的普遍性以及安全性,尤其关注物联网设备的安全性。
	
	最近的一些工作建立在网络扫描的基础之上,已经使得对大量设备的识别成为可能。Feng等人构建了一个能够利用应用层响应而无需机器学习来进行设备识别的系统,这和前面提到的我们手工编写的专家规则类似。[21]。这项工作促成了许多利用banners和其他主机信息来描述主机特征的论文。[6,20,39,50,51]。还有其他一些基于规则的引擎已经被用于另外一些基于探测应用banners的主动的公共扫描数据的工作中。
	
	\subsubsection*{家庭物联网安全}
	鉴于其持续增长的从系统级到应用层的安全和隐私需求,家庭物联网安全在最近已经成了一个研究热点。Ma等人调查了Mirai僵尸网络的兴起——由大量的因为弱凭证被入侵的物联网设备组成用来发起大规模DDOS攻击的网络。其实不局限于恶意攻击者,从物联网设备诞生以来研究人员就在一直实施相关攻击(出于研究目的)[8,31,38,47,59]。最近Fernandes等人也从访问控制策略到第三方开发者集成概述了三星SmartThings(三星的智能家居设备平台)设备面临的一系列挑战[22]。为此,研究人员建立了一套系统来实现家庭物联网的安全属性,如信息流跟踪和沙盒化[23,33],改进设备身份认证方式[54],以及启用审计信息[55,58]。最近,Alrawi等人将家庭物联网设备的安全性做了一个整合,其中他们系统地提出了对家庭物联网的攻击和防御手段,并概述了该如何评估家庭物联网的风险[2]。
	
	\subsubsection*{互联网范围的物联网设备扫描}
	最近有大量的使用互联网范围内的扫描进行安全分析的工作,包括分析公共网络下的嵌入式系统(例如,[4、7、18、24、28、30、36、37、40、48、52、60])。相比之下,我们的工作则着重于家庭内部的设备——这些设备在互联网范围的扫描下是不可见的。
	
\end{document}